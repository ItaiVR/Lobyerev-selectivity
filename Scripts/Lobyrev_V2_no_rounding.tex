\documentclass[]{article}
\usepackage{lmodern}
\usepackage{amssymb,amsmath}
\usepackage{ifxetex,ifluatex}
\usepackage{fixltx2e} % provides \textsubscript
\ifnum 0\ifxetex 1\fi\ifluatex 1\fi=0 % if pdftex
  \usepackage[T1]{fontenc}
  \usepackage[utf8]{inputenc}
\else % if luatex or xelatex
  \ifxetex
    \usepackage{mathspec}
  \else
    \usepackage{fontspec}
  \fi
  \defaultfontfeatures{Ligatures=TeX,Scale=MatchLowercase}
\fi
% use upquote if available, for straight quotes in verbatim environments
\IfFileExists{upquote.sty}{\usepackage{upquote}}{}
% use microtype if available
\IfFileExists{microtype.sty}{%
\usepackage{microtype}
\UseMicrotypeSet[protrusion]{basicmath} % disable protrusion for tt fonts
}{}
\usepackage[margin=1in]{geometry}
\usepackage{hyperref}
\hypersetup{unicode=true,
            pdftitle={Lobyerev V1},
            pdfauthor={Itai van Rijn},
            pdfborder={0 0 0},
            breaklinks=true}
\urlstyle{same}  % don't use monospace font for urls
\usepackage{color}
\usepackage{fancyvrb}
\newcommand{\VerbBar}{|}
\newcommand{\VERB}{\Verb[commandchars=\\\{\}]}
\DefineVerbatimEnvironment{Highlighting}{Verbatim}{commandchars=\\\{\}}
% Add ',fontsize=\small' for more characters per line
\usepackage{framed}
\definecolor{shadecolor}{RGB}{248,248,248}
\newenvironment{Shaded}{\begin{snugshade}}{\end{snugshade}}
\newcommand{\KeywordTok}[1]{\textcolor[rgb]{0.13,0.29,0.53}{\textbf{#1}}}
\newcommand{\DataTypeTok}[1]{\textcolor[rgb]{0.13,0.29,0.53}{#1}}
\newcommand{\DecValTok}[1]{\textcolor[rgb]{0.00,0.00,0.81}{#1}}
\newcommand{\BaseNTok}[1]{\textcolor[rgb]{0.00,0.00,0.81}{#1}}
\newcommand{\FloatTok}[1]{\textcolor[rgb]{0.00,0.00,0.81}{#1}}
\newcommand{\ConstantTok}[1]{\textcolor[rgb]{0.00,0.00,0.00}{#1}}
\newcommand{\CharTok}[1]{\textcolor[rgb]{0.31,0.60,0.02}{#1}}
\newcommand{\SpecialCharTok}[1]{\textcolor[rgb]{0.00,0.00,0.00}{#1}}
\newcommand{\StringTok}[1]{\textcolor[rgb]{0.31,0.60,0.02}{#1}}
\newcommand{\VerbatimStringTok}[1]{\textcolor[rgb]{0.31,0.60,0.02}{#1}}
\newcommand{\SpecialStringTok}[1]{\textcolor[rgb]{0.31,0.60,0.02}{#1}}
\newcommand{\ImportTok}[1]{#1}
\newcommand{\CommentTok}[1]{\textcolor[rgb]{0.56,0.35,0.01}{\textit{#1}}}
\newcommand{\DocumentationTok}[1]{\textcolor[rgb]{0.56,0.35,0.01}{\textbf{\textit{#1}}}}
\newcommand{\AnnotationTok}[1]{\textcolor[rgb]{0.56,0.35,0.01}{\textbf{\textit{#1}}}}
\newcommand{\CommentVarTok}[1]{\textcolor[rgb]{0.56,0.35,0.01}{\textbf{\textit{#1}}}}
\newcommand{\OtherTok}[1]{\textcolor[rgb]{0.56,0.35,0.01}{#1}}
\newcommand{\FunctionTok}[1]{\textcolor[rgb]{0.00,0.00,0.00}{#1}}
\newcommand{\VariableTok}[1]{\textcolor[rgb]{0.00,0.00,0.00}{#1}}
\newcommand{\ControlFlowTok}[1]{\textcolor[rgb]{0.13,0.29,0.53}{\textbf{#1}}}
\newcommand{\OperatorTok}[1]{\textcolor[rgb]{0.81,0.36,0.00}{\textbf{#1}}}
\newcommand{\BuiltInTok}[1]{#1}
\newcommand{\ExtensionTok}[1]{#1}
\newcommand{\PreprocessorTok}[1]{\textcolor[rgb]{0.56,0.35,0.01}{\textit{#1}}}
\newcommand{\AttributeTok}[1]{\textcolor[rgb]{0.77,0.63,0.00}{#1}}
\newcommand{\RegionMarkerTok}[1]{#1}
\newcommand{\InformationTok}[1]{\textcolor[rgb]{0.56,0.35,0.01}{\textbf{\textit{#1}}}}
\newcommand{\WarningTok}[1]{\textcolor[rgb]{0.56,0.35,0.01}{\textbf{\textit{#1}}}}
\newcommand{\AlertTok}[1]{\textcolor[rgb]{0.94,0.16,0.16}{#1}}
\newcommand{\ErrorTok}[1]{\textcolor[rgb]{0.64,0.00,0.00}{\textbf{#1}}}
\newcommand{\NormalTok}[1]{#1}
\usepackage{graphicx,grffile}
\makeatletter
\def\maxwidth{\ifdim\Gin@nat@width>\linewidth\linewidth\else\Gin@nat@width\fi}
\def\maxheight{\ifdim\Gin@nat@height>\textheight\textheight\else\Gin@nat@height\fi}
\makeatother
% Scale images if necessary, so that they will not overflow the page
% margins by default, and it is still possible to overwrite the defaults
% using explicit options in \includegraphics[width, height, ...]{}
\setkeys{Gin}{width=\maxwidth,height=\maxheight,keepaspectratio}
\IfFileExists{parskip.sty}{%
\usepackage{parskip}
}{% else
\setlength{\parindent}{0pt}
\setlength{\parskip}{6pt plus 2pt minus 1pt}
}
\setlength{\emergencystretch}{3em}  % prevent overfull lines
\providecommand{\tightlist}{%
  \setlength{\itemsep}{0pt}\setlength{\parskip}{0pt}}
\setcounter{secnumdepth}{0}
% Redefines (sub)paragraphs to behave more like sections
\ifx\paragraph\undefined\else
\let\oldparagraph\paragraph
\renewcommand{\paragraph}[1]{\oldparagraph{#1}\mbox{}}
\fi
\ifx\subparagraph\undefined\else
\let\oldsubparagraph\subparagraph
\renewcommand{\subparagraph}[1]{\oldsubparagraph{#1}\mbox{}}
\fi

%%% Use protect on footnotes to avoid problems with footnotes in titles
\let\rmarkdownfootnote\footnote%
\def\footnote{\protect\rmarkdownfootnote}

%%% Change title format to be more compact
\usepackage{titling}

% Create subtitle command for use in maketitle
\newcommand{\subtitle}[1]{
  \posttitle{
    \begin{center}\large#1\end{center}
    }
}

\setlength{\droptitle}{-2em}

  \title{Lobyerev V1}
    \pretitle{\vspace{\droptitle}\centering\huge}
  \posttitle{\par}
    \author{Itai van Rijn}
    \preauthor{\centering\large\emph}
  \postauthor{\par}
      \predate{\centering\large\emph}
  \postdate{\par}
    \date{February 24, 2019}


\begin{document}
\maketitle

\subsection{implementation of Lobyerev \& Hoffman (2008) Selectivity
model}\label{implementation-of-lobyerev-hoffman-2008-selectivity-model}

This document describes the implementation of the selectivity algorithm
described in Lobyrev, Feodor, and Matthew J. Hoffman. ``A morphological
and geometric method for estimating the selectivity of gill nets.''
Reviews in Fish Biology and Fisheries 28.4 (2018): 909-924.

\subsubsection{Step 1: Import field data on fish catch and gill net
properties}\label{step-1-import-field-data-on-fish-catch-and-gill-net-properties}

\paragraph{Step 1.1: Import catch
data}\label{step-1.1-import-catch-data}

Catch data structure: Column 1: `Mesh\_size' - name of the net (later
connected to net properties under the same name) Column 2:
`Length\_group' - Fish total length (cm) Column 3: `Wedged' - Number of
wedged individuals Column 4: `Tangled' - Number of tangled individuals

\begin{Shaded}
\begin{Highlighting}[]
\NormalTok{Catch.data.Cod <-}\StringTok{ }\KeywordTok{read.csv}\NormalTok{(}\StringTok{"~/kinneret modeling/selectivity/R code Feodor paper/Data/Catch data Cod.csv"}\NormalTok{)}
\CommentTok{#Show the first 6 lines:}
\KeywordTok{print}\NormalTok{(}\KeywordTok{head}\NormalTok{(Catch.data.Cod))}
\end{Highlighting}
\end{Shaded}

\begin{verbatim}
##   Mesh_size Length_group Wedged Tangled
## 1        20           14      2       2
## 2        20           16     20       5
## 3        20           18     15      12
## 4        20           20     11       8
## 5        20           22     20      18
## 6        20           24     32      24
\end{verbatim}

\paragraph{Step 1.2: Import table of net
properties}\label{step-1.2-import-table-of-net-properties}

Net properties data structure: (See Figure 2 in paper) Column 1:
`Mesh\_size' - mesh size (knot to knot) (mm) Column 2: `y' - Smaller of
the angels between mesh threads

\begin{Shaded}
\begin{Highlighting}[]
\NormalTok{net.properties <-}\StringTok{ }\KeywordTok{read.csv}\NormalTok{(}\StringTok{"~/kinneret modeling/selectivity/R code Feodor paper/Data/net properties.csv"}\NormalTok{)}
\CommentTok{#Show table:}
\KeywordTok{print}\NormalTok{(net.properties)}
\end{Highlighting}
\end{Shaded}

\begin{verbatim}
##   Mesh_size  y
## 1        20 60
## 2        25 60
## 3        30 60
\end{verbatim}

\paragraph{Step 1.3: Add angel in
Radian}\label{step-1.3-add-angel-in-radian}

\begin{Shaded}
\begin{Highlighting}[]
\NormalTok{net.properties}\OperatorTok{$}\NormalTok{Radian=}\FloatTok{0.018}\OperatorTok{*}\NormalTok{net.properties}\OperatorTok{$}\NormalTok{y}
\CommentTok{#Show table:}
\KeywordTok{print}\NormalTok{(net.properties)}
\end{Highlighting}
\end{Shaded}

\begin{verbatim}
##   Mesh_size  y Radian
## 1        20 60   1.08
## 2        25 60   1.08
## 3        30 60   1.08
\end{verbatim}

\paragraph{Step 1.4: Define the angel between the end of upper and lower
jaws
(phi)}\label{step-1.4-define-the-angel-between-the-end-of-upper-and-lower-jaws-phi}

\begin{Shaded}
\begin{Highlighting}[]
\NormalTok{phi_deg=}\DecValTok{7}
\NormalTok{phi_radian=}\FloatTok{0.018}\OperatorTok{*}\NormalTok{phi_deg}
\end{Highlighting}
\end{Shaded}

\paragraph{Step 1.5: Calculate the jaw length approximated by the linear
function}\label{step-1.5-calculate-the-jaw-length-approximated-by-the-linear-function}

\begin{Shaded}
\begin{Highlighting}[]
\NormalTok{slope_jaw=}\FloatTok{1.02}
\NormalTok{intecept_jaw=}\FloatTok{3.41}
\NormalTok{Catch.data.Cod}\OperatorTok{$}\NormalTok{Jaw_length=intecept_jaw }\OperatorTok{+}\StringTok{ }\NormalTok{(slope_jaw}\OperatorTok{*}\NormalTok{Catch.data.Cod}\OperatorTok{$}\NormalTok{Length_group)}
\CommentTok{#Calculate h}
\NormalTok{Catch.data.Cod}\OperatorTok{$}\NormalTok{h=Catch.data.Cod}\OperatorTok{$}\NormalTok{Jaw_length }\OperatorTok{*}\StringTok{ }\KeywordTok{sin}\NormalTok{(phi_radian) }\OperatorTok{*}\StringTok{ }\DecValTok{2}
\KeywordTok{print}\NormalTok{(}\KeywordTok{head}\NormalTok{(Catch.data.Cod))}
\end{Highlighting}
\end{Shaded}

\begin{verbatim}
##   Mesh_size Length_group Wedged Tangled Jaw_length        h
## 1        20           14      2       2      17.69 4.446094
## 2        20           16     20       5      19.73 4.958815
## 3        20           18     15      12      21.77 5.471535
## 4        20           20     11       8      23.81 5.984256
## 5        20           22     20      18      25.85 6.496977
## 6        20           24     32      24      27.89 7.009698
\end{verbatim}

\paragraph{Step 1.6: Merge the catch data with the net
data}\label{step-1.6-merge-the-catch-data-with-the-net-data}

\begin{Shaded}
\begin{Highlighting}[]
\NormalTok{Catch.data.Cod=}\KeywordTok{merge}\NormalTok{(Catch.data.Cod,net.properties,}\DataTypeTok{by=}\StringTok{"Mesh_size"}\NormalTok{)}
\CommentTok{#print first 6 lines}
\KeywordTok{print}\NormalTok{(}\KeywordTok{head}\NormalTok{(Catch.data.Cod))}
\end{Highlighting}
\end{Shaded}

\begin{verbatim}
##   Mesh_size Length_group Wedged Tangled Jaw_length        h  y Radian
## 1        20           14      2       2      17.69 4.446094 60   1.08
## 2        20           16     20       5      19.73 4.958815 60   1.08
## 3        20           18     15      12      21.77 5.471535 60   1.08
## 4        20           20     11       8      23.81 5.984256 60   1.08
## 5        20           22     20      18      25.85 6.496977 60   1.08
## 6        20           24     32      24      27.89 7.009698 60   1.08
\end{verbatim}

\subsubsection{Step 2: Calculate P(O\textbar{}C) and
P(Th\textbar{}C)}\label{step-2-calculate-poc-and-pthc}

For calculating P(O\textbar{}C) use eq. 4 in the paper

\begin{Shaded}
\begin{Highlighting}[]
\CommentTok{#Calculate sin,cos,tan}
\NormalTok{Catch.data.Cod}\OperatorTok{$}\NormalTok{sin=}\KeywordTok{sin}\NormalTok{(}\FloatTok{0.5}\OperatorTok{*}\NormalTok{Catch.data.Cod}\OperatorTok{$}\NormalTok{Radian)}
\NormalTok{Catch.data.Cod}\OperatorTok{$}\NormalTok{cos=}\KeywordTok{cos}\NormalTok{(}\FloatTok{0.5}\OperatorTok{*}\NormalTok{Catch.data.Cod}\OperatorTok{$}\NormalTok{Radian)}
\NormalTok{Catch.data.Cod}\OperatorTok{$}\NormalTok{tan=}\KeywordTok{tan}\NormalTok{(}\FloatTok{0.5}\OperatorTok{*}\NormalTok{Catch.data.Cod}\OperatorTok{$}\NormalTok{Radian)}
\CommentTok{#Calculate P(O|C) Eq. 4}
\NormalTok{Catch.data.Cod}\OperatorTok{$}\NormalTok{POC=(((Catch.data.Cod}\OperatorTok{$}\NormalTok{cos }\OperatorTok{*}\StringTok{ }\NormalTok{Catch.data.Cod}\OperatorTok{$}\NormalTok{Mesh_size)}\OperatorTok{-}\NormalTok{Catch.data.Cod}\OperatorTok{$}\NormalTok{h)}\OperatorTok{*}\NormalTok{((Catch.data.Cod}\OperatorTok{$}\NormalTok{sin}\OperatorTok{*}\NormalTok{Catch.data.Cod}\OperatorTok{$}\NormalTok{Mesh_size)}\OperatorTok{-}\NormalTok{(Catch.data.Cod}\OperatorTok{$}\NormalTok{tan}\OperatorTok{*}\NormalTok{Catch.data.Cod}\OperatorTok{$}\NormalTok{h)))}\OperatorTok{/}\NormalTok{(Catch.data.Cod}\OperatorTok{$}\NormalTok{cos}\OperatorTok{*}\NormalTok{Catch.data.Cod}\OperatorTok{$}\NormalTok{sin}\OperatorTok{*}\NormalTok{(Catch.data.Cod}\OperatorTok{$}\NormalTok{Mesh_size)}\OperatorTok{^}\DecValTok{2}\NormalTok{)}
\CommentTok{#Calculate P(Th|c)}
\NormalTok{Catch.data.Cod}\OperatorTok{$}\NormalTok{PThC=}\DecValTok{1}\OperatorTok{-}\NormalTok{Catch.data.Cod}\OperatorTok{$}\NormalTok{POC}
\CommentTok{#Print first 6 rows}
\KeywordTok{print}\NormalTok{(}\KeywordTok{head}\NormalTok{(Catch.data.Cod))}
\end{Highlighting}
\end{Shaded}

\begin{verbatim}
##   Mesh_size Length_group Wedged Tangled Jaw_length        h  y Radian
## 1        20           14      2       2      17.69 4.446094 60   1.08
## 2        20           16     20       5      19.73 4.958815 60   1.08
## 3        20           18     15      12      21.77 5.471535 60   1.08
## 4        20           20     11       8      23.81 5.984256 60   1.08
## 5        20           22     20      18      25.85 6.496977 60   1.08
## 6        20           24     32      24      27.89 7.009698 60   1.08
##        sin       cos       tan       POC      PThC
## 1 0.514136 0.8577087 0.5994296 0.5488078 0.4511922
## 2 0.514136 0.8577087 0.5994296 0.5054167 0.4945833
## 3 0.514136 0.8577087 0.5994296 0.4638123 0.5361877
## 4 0.514136 0.8577087 0.5994296 0.4239946 0.5760054
## 5 0.514136 0.8577087 0.5994296 0.3859636 0.6140364
## 6 0.514136 0.8577087 0.5994296 0.3497193 0.6502807
\end{verbatim}

\subsubsection{Step 3: Calculate
P(W\textbar{}E)}\label{step-3-calculate-pwe}

!!!!I just import the data from your Excel, but do not know how the
calculation was made

\begin{Shaded}
\begin{Highlighting}[]
\NormalTok{PWE <-}\StringTok{ }\KeywordTok{read.csv}\NormalTok{(}\StringTok{"~/kinneret modeling/selectivity/R code Feodor paper/Data/PWE.csv"}\NormalTok{, }\DataTypeTok{stringsAsFactors=}\OtherTok{FALSE}\NormalTok{)}
\CommentTok{#print the first 6 lines}
\KeywordTok{print}\NormalTok{(}\KeywordTok{head}\NormalTok{(PWE))}
\end{Highlighting}
\end{Shaded}

\begin{verbatim}
##   Mesh_size Length_group  PWE
## 1        20           14 0.35
## 2        20           16 0.69
## 3        20           18 0.81
## 4        20           20 0.88
## 5        20           22 0.88
## 6        20           24 1.00
\end{verbatim}

\begin{Shaded}
\begin{Highlighting}[]
\CommentTok{#merge with table 'Catch.data.Cod'}
\NormalTok{Catch.data.Cod=}\KeywordTok{merge}\NormalTok{(Catch.data.Cod,PWE ,}\DataTypeTok{by=}\KeywordTok{c}\NormalTok{(}\StringTok{"Mesh_size"}\NormalTok{,    }\StringTok{"Length_group"}\NormalTok{))}
\end{Highlighting}
\end{Shaded}

\subsubsection{Step 4: Calculate Eq. 2}\label{step-4-calculate-eq.-2}

\paragraph{Step 4.1: Calculate P(E\textbar{}O) by the linear
equation}\label{step-4.1-calculate-peo-by-the-linear-equation}

\begin{Shaded}
\begin{Highlighting}[]
\NormalTok{##Get the min and max size groups for each net}
\CommentTok{#Create table}
\NormalTok{net_PEO=}\KeywordTok{data.frame}\NormalTok{(}\DataTypeTok{Mesh_size=}\KeywordTok{unique}\NormalTok{(Catch.data.Cod}\OperatorTok{$}\NormalTok{Mesh_size))}
\CommentTok{#Length group interval}
\NormalTok{Length_group_interval=}\DecValTok{2}
\CommentTok{#The minimal size for each net}
\NormalTok{min_wedged=}\KeywordTok{data.frame}\NormalTok{(Catch.data.Cod }\OperatorTok
\StringTok{   }\KeywordTok{filter}\NormalTok{(}\OperatorTok{!}\KeywordTok{is.na}\NormalTok{(Wedged)) }\OperatorTok
\StringTok{  }\KeywordTok{group_by}\NormalTok{(Mesh_size) }\OperatorTok
\StringTok{  }\KeywordTok{summarize}\NormalTok{(}\DataTypeTok{min_size =} \KeywordTok{min}\NormalTok{(Length_group, }\DataTypeTok{na.rm =} \OtherTok{TRUE}\NormalTok{)))}
\CommentTok{#The maximal size for each net}
\NormalTok{max_wedged=}\KeywordTok{data.frame}\NormalTok{(Catch.data.Cod }\OperatorTok
\StringTok{   }\KeywordTok{filter}\NormalTok{(}\OperatorTok{!}\KeywordTok{is.na}\NormalTok{(Wedged)) }\OperatorTok
\StringTok{  }\KeywordTok{group_by}\NormalTok{(Mesh_size) }\OperatorTok
\StringTok{  }\KeywordTok{summarize}\NormalTok{(}\DataTypeTok{max_size =} \KeywordTok{max}\NormalTok{(Length_group, }\DataTypeTok{na.rm =} \OtherTok{TRUE}\NormalTok{)))}
\CommentTok{#merge}
\NormalTok{net_PEO=}\KeywordTok{merge}\NormalTok{(net_PEO,min_wedged,}\DataTypeTok{by=}\StringTok{"Mesh_size"}\NormalTok{)}
\NormalTok{net_PEO=}\KeywordTok{merge}\NormalTok{(net_PEO,max_wedged,}\DataTypeTok{by=}\StringTok{"Mesh_size"}\NormalTok{)}
\CommentTok{#Substract and add the Length group interval}
\NormalTok{net_PEO}\OperatorTok{$}\NormalTok{min_size=net_PEO}\OperatorTok{$}\NormalTok{min_size}\OperatorTok{-}\NormalTok{(Length_group_interval}\OperatorTok{/}\DecValTok{2}\NormalTok{)}
\NormalTok{net_PEO}\OperatorTok{$}\NormalTok{max_size=net_PEO}\OperatorTok{$}\NormalTok{max_size}\OperatorTok{+}\NormalTok{(Length_group_interval}\OperatorTok{/}\DecValTok{2}\NormalTok{)}
\CommentTok{#Calculate the linear function}
\NormalTok{net_PEO}\OperatorTok{$}\NormalTok{slope=}\OperatorTok{-}\DecValTok{1}\OperatorTok{/}\NormalTok{(net_PEO}\OperatorTok{$}\NormalTok{max_size}\OperatorTok{-}\NormalTok{net_PEO}\OperatorTok{$}\NormalTok{min_size)}
\NormalTok{net_PEO}\OperatorTok{$}\NormalTok{intercept=}\DecValTok{1}\OperatorTok{-}\NormalTok{(net_PEO}\OperatorTok{$}\NormalTok{slope}\OperatorTok{*}\NormalTok{net_PEO}\OperatorTok{$}\NormalTok{min_size)}
\CommentTok{#print}
\KeywordTok{print}\NormalTok{(net_PEO)}
\end{Highlighting}
\end{Shaded}

\begin{verbatim}
##   Mesh_size min_size max_size       slope intercept
## 1        20       13       31 -0.05555556  1.722222
## 2        25       17       35 -0.05555556  1.944444
## 3        30       21       37 -0.06250000  2.312500
\end{verbatim}

\begin{Shaded}
\begin{Highlighting}[]
\CommentTok{#Calculate PEO}
\CommentTok{#Merge slope and intercept to data}
\NormalTok{Catch.data.Cod=}\KeywordTok{merge}\NormalTok{(Catch.data.Cod,net_PEO,}\DataTypeTok{by=}\StringTok{"Mesh_size"}\NormalTok{)}
\NormalTok{Catch.data.Cod}\OperatorTok{$}\NormalTok{PEO=Catch.data.Cod}\OperatorTok{$}\NormalTok{intercept}\OperatorTok{+}\NormalTok{(Catch.data.Cod}\OperatorTok{$}\NormalTok{Length_group)}\OperatorTok{*}\NormalTok{Catch.data.Cod}\OperatorTok{$}\NormalTok{slope}
\CommentTok{#Remove PEO values if no fish were wedged}
\NormalTok{Catch.data.Cod[}\KeywordTok{is.na}\NormalTok{(Catch.data.Cod}\OperatorTok{$}\NormalTok{Wedged),}\StringTok{"PEO"}\NormalTok{]=}\OtherTok{NA}
\CommentTok{#Print first 6 lines}
\KeywordTok{print}\NormalTok{(}\KeywordTok{head}\NormalTok{(Catch.data.Cod))}
\end{Highlighting}
\end{Shaded}

\begin{verbatim}
##   Mesh_size Length_group Wedged Tangled Jaw_length        h  y Radian
## 1        20           14      2       2      17.69 4.446094 60   1.08
## 2        20           16     20       5      19.73 4.958815 60   1.08
## 3        20           18     15      12      21.77 5.471535 60   1.08
## 4        20           20     11       8      23.81 5.984256 60   1.08
## 5        20           22     20      18      25.85 6.496977 60   1.08
## 6        20           24     32      24      27.89 7.009698 60   1.08
##        sin       cos       tan       POC      PThC  PWE min_size max_size
## 1 0.514136 0.8577087 0.5994296 0.5488078 0.4511922 0.35       13       31
## 2 0.514136 0.8577087 0.5994296 0.5054167 0.4945833 0.69       13       31
## 3 0.514136 0.8577087 0.5994296 0.4638123 0.5361877 0.81       13       31
## 4 0.514136 0.8577087 0.5994296 0.4239946 0.5760054 0.88       13       31
## 5 0.514136 0.8577087 0.5994296 0.3859636 0.6140364 0.88       13       31
## 6 0.514136 0.8577087 0.5994296 0.3497193 0.6502807 1.00       13       31
##         slope intercept       PEO
## 1 -0.05555556  1.722222 0.9444444
## 2 -0.05555556  1.722222 0.8333333
## 3 -0.05555556  1.722222 0.7222222
## 4 -0.05555556  1.722222 0.6111111
## 5 -0.05555556  1.722222 0.5000000
## 6 -0.05555556  1.722222 0.3888889
\end{verbatim}

\paragraph{Step 4.2: Claculate Eq.2}\label{step-4.2-claculate-eq.2}

\begin{Shaded}
\begin{Highlighting}[]
\NormalTok{Catch.data.Cod}\OperatorTok{$}\NormalTok{Ntotal=Catch.data.Cod}\OperatorTok{$}\NormalTok{Wedged}\OperatorTok{/}\NormalTok{(Catch.data.Cod}\OperatorTok{$}\NormalTok{POC}\OperatorTok{*}\NormalTok{Catch.data.Cod}\OperatorTok{$}\NormalTok{PWE}\OperatorTok{*}\NormalTok{Catch.data.Cod}\OperatorTok{$}\NormalTok{PEO)}
\CommentTok{#Print first 6 lines}
\KeywordTok{print}\NormalTok{(}\KeywordTok{head}\NormalTok{(Catch.data.Cod))}
\end{Highlighting}
\end{Shaded}

\begin{verbatim}
##   Mesh_size Length_group Wedged Tangled Jaw_length        h  y Radian
## 1        20           14      2       2      17.69 4.446094 60   1.08
## 2        20           16     20       5      19.73 4.958815 60   1.08
## 3        20           18     15      12      21.77 5.471535 60   1.08
## 4        20           20     11       8      23.81 5.984256 60   1.08
## 5        20           22     20      18      25.85 6.496977 60   1.08
## 6        20           24     32      24      27.89 7.009698 60   1.08
##        sin       cos       tan       POC      PThC  PWE min_size max_size
## 1 0.514136 0.8577087 0.5994296 0.5488078 0.4511922 0.35       13       31
## 2 0.514136 0.8577087 0.5994296 0.5054167 0.4945833 0.69       13       31
## 3 0.514136 0.8577087 0.5994296 0.4638123 0.5361877 0.81       13       31
## 4 0.514136 0.8577087 0.5994296 0.4239946 0.5760054 0.88       13       31
## 5 0.514136 0.8577087 0.5994296 0.3859636 0.6140364 0.88       13       31
## 6 0.514136 0.8577087 0.5994296 0.3497193 0.6502807 1.00       13       31
##         slope intercept       PEO    Ntotal
## 1 -0.05555556  1.722222 0.9444444  11.02466
## 2 -0.05555556  1.722222 0.8333333  68.81967
## 3 -0.05555556  1.722222 0.7222222  55.28320
## 4 -0.05555556  1.722222 0.6111111  48.24247
## 5 -0.05555556  1.722222 0.5000000 117.76900
## 6 -0.05555556  1.722222 0.3888889 235.29074
\end{verbatim}

\subsubsection{Step 5: Calculate CPUE}\label{step-5-calculate-cpue}

(table 21)

\begin{Shaded}
\begin{Highlighting}[]
\CommentTok{#Define the number of trials}
\NormalTok{n_trials=}\DecValTok{11}
\CommentTok{#Aggregate catch for each mesh size}
\NormalTok{Catch.data.Cod}\OperatorTok{$}\NormalTok{CPUE=}\KeywordTok{rowSums}\NormalTok{(Catch.data.Cod[,}\KeywordTok{c}\NormalTok{(}\StringTok{"Wedged"}\NormalTok{,}\StringTok{"Tangled"}\NormalTok{)],}\DataTypeTok{na.rm=}\NormalTok{T)}\OperatorTok{/}\NormalTok{n_trials}
\CommentTok{#Print first 6 lines}
\KeywordTok{print}\NormalTok{(}\KeywordTok{head}\NormalTok{(Catch.data.Cod))}
\end{Highlighting}
\end{Shaded}

\begin{verbatim}
##   Mesh_size Length_group Wedged Tangled Jaw_length        h  y Radian
## 1        20           14      2       2      17.69 4.446094 60   1.08
## 2        20           16     20       5      19.73 4.958815 60   1.08
## 3        20           18     15      12      21.77 5.471535 60   1.08
## 4        20           20     11       8      23.81 5.984256 60   1.08
## 5        20           22     20      18      25.85 6.496977 60   1.08
## 6        20           24     32      24      27.89 7.009698 60   1.08
##        sin       cos       tan       POC      PThC  PWE min_size max_size
## 1 0.514136 0.8577087 0.5994296 0.5488078 0.4511922 0.35       13       31
## 2 0.514136 0.8577087 0.5994296 0.5054167 0.4945833 0.69       13       31
## 3 0.514136 0.8577087 0.5994296 0.4638123 0.5361877 0.81       13       31
## 4 0.514136 0.8577087 0.5994296 0.4239946 0.5760054 0.88       13       31
## 5 0.514136 0.8577087 0.5994296 0.3859636 0.6140364 0.88       13       31
## 6 0.514136 0.8577087 0.5994296 0.3497193 0.6502807 1.00       13       31
##         slope intercept       PEO    Ntotal      CPUE
## 1 -0.05555556  1.722222 0.9444444  11.02466 0.3636364
## 2 -0.05555556  1.722222 0.8333333  68.81967 2.2727273
## 3 -0.05555556  1.722222 0.7222222  55.28320 2.4545455
## 4 -0.05555556  1.722222 0.6111111  48.24247 1.7272727
## 5 -0.05555556  1.722222 0.5000000 117.76900 3.4545455
## 6 -0.05555556  1.722222 0.3888889 235.29074 5.0909091
\end{verbatim}

\subsubsection{Step 6: Calculate Nw per
hour}\label{step-6-calculate-nw-per-hour}

(table 23)

\begin{Shaded}
\begin{Highlighting}[]
\CommentTok{#Define number of hours of single (??) field trial}
\NormalTok{n_hours=}\DecValTok{12}
\NormalTok{Catch.data.Cod}\OperatorTok{$}\NormalTok{Ntotal_per_hour=Catch.data.Cod}\OperatorTok{$}\NormalTok{Ntotal}\OperatorTok{/}\NormalTok{n_hours}
\CommentTok{#Print first 6 lines}
\KeywordTok{print}\NormalTok{(}\KeywordTok{head}\NormalTok{(Catch.data.Cod))}
\end{Highlighting}
\end{Shaded}

\begin{verbatim}
##   Mesh_size Length_group Wedged Tangled Jaw_length        h  y Radian
## 1        20           14      2       2      17.69 4.446094 60   1.08
## 2        20           16     20       5      19.73 4.958815 60   1.08
## 3        20           18     15      12      21.77 5.471535 60   1.08
## 4        20           20     11       8      23.81 5.984256 60   1.08
## 5        20           22     20      18      25.85 6.496977 60   1.08
## 6        20           24     32      24      27.89 7.009698 60   1.08
##        sin       cos       tan       POC      PThC  PWE min_size max_size
## 1 0.514136 0.8577087 0.5994296 0.5488078 0.4511922 0.35       13       31
## 2 0.514136 0.8577087 0.5994296 0.5054167 0.4945833 0.69       13       31
## 3 0.514136 0.8577087 0.5994296 0.4638123 0.5361877 0.81       13       31
## 4 0.514136 0.8577087 0.5994296 0.4239946 0.5760054 0.88       13       31
## 5 0.514136 0.8577087 0.5994296 0.3859636 0.6140364 0.88       13       31
## 6 0.514136 0.8577087 0.5994296 0.3497193 0.6502807 1.00       13       31
##         slope intercept       PEO    Ntotal      CPUE Ntotal_per_hour
## 1 -0.05555556  1.722222 0.9444444  11.02466 0.3636364       0.9187218
## 2 -0.05555556  1.722222 0.8333333  68.81967 2.2727273       5.7349724
## 3 -0.05555556  1.722222 0.7222222  55.28320 2.4545455       4.6069330
## 4 -0.05555556  1.722222 0.6111111  48.24247 1.7272727       4.0202057
## 5 -0.05555556  1.722222 0.5000000 117.76900 3.4545455       9.8140830
## 6 -0.05555556  1.722222 0.3888889 235.29074 5.0909091      19.6075616
\end{verbatim}

\subsubsection{Step 7:}\label{step-7}

\paragraph{Step 7.1: SLl,t for each length class as
Ntotal-Qt}\label{step-7.1-sllt-for-each-length-class-as-ntotal-qt}

(table 24)

\begin{Shaded}
\begin{Highlighting}[]
\NormalTok{Catch.data.Cod}\OperatorTok{$}\NormalTok{SL_l_t=Catch.data.Cod}\OperatorTok{$}\NormalTok{Ntotal_per_hour}\OperatorTok{-}\NormalTok{Catch.data.Cod}\OperatorTok{$}\NormalTok{CPUE}
\CommentTok{#Print first 6 lines}
\KeywordTok{print}\NormalTok{(}\KeywordTok{head}\NormalTok{(Catch.data.Cod))}
\end{Highlighting}
\end{Shaded}

\begin{verbatim}
##   Mesh_size Length_group Wedged Tangled Jaw_length        h  y Radian
## 1        20           14      2       2      17.69 4.446094 60   1.08
## 2        20           16     20       5      19.73 4.958815 60   1.08
## 3        20           18     15      12      21.77 5.471535 60   1.08
## 4        20           20     11       8      23.81 5.984256 60   1.08
## 5        20           22     20      18      25.85 6.496977 60   1.08
## 6        20           24     32      24      27.89 7.009698 60   1.08
##        sin       cos       tan       POC      PThC  PWE min_size max_size
## 1 0.514136 0.8577087 0.5994296 0.5488078 0.4511922 0.35       13       31
## 2 0.514136 0.8577087 0.5994296 0.5054167 0.4945833 0.69       13       31
## 3 0.514136 0.8577087 0.5994296 0.4638123 0.5361877 0.81       13       31
## 4 0.514136 0.8577087 0.5994296 0.4239946 0.5760054 0.88       13       31
## 5 0.514136 0.8577087 0.5994296 0.3859636 0.6140364 0.88       13       31
## 6 0.514136 0.8577087 0.5994296 0.3497193 0.6502807 1.00       13       31
##         slope intercept       PEO    Ntotal      CPUE Ntotal_per_hour
## 1 -0.05555556  1.722222 0.9444444  11.02466 0.3636364       0.9187218
## 2 -0.05555556  1.722222 0.8333333  68.81967 2.2727273       5.7349724
## 3 -0.05555556  1.722222 0.7222222  55.28320 2.4545455       4.6069330
## 4 -0.05555556  1.722222 0.6111111  48.24247 1.7272727       4.0202057
## 5 -0.05555556  1.722222 0.5000000 117.76900 3.4545455       9.8140830
## 6 -0.05555556  1.722222 0.3888889 235.29074 5.0909091      19.6075616
##       SL_l_t
## 1  0.5550854
## 2  3.4622451
## 3  2.1523876
## 4  2.2929329
## 5  6.3595376
## 6 14.5166525
\end{verbatim}

\paragraph{Step 7.2: N\_AP calculation}\label{step-7.2-n_ap-calculation}

First the table of Nlim (table 22c) is defined, in reality it is in
input

!!!Notice- value of tau is very sensitive to the sum of the CPUE

\begin{Shaded}
\begin{Highlighting}[]
\NormalTok{###The next data frame will be an experimental input}
\NormalTok{n_lim=}\KeywordTok{data.frame}\NormalTok{(}\DataTypeTok{Mesh_size=}\KeywordTok{c}\NormalTok{(}\DecValTok{20}\NormalTok{,}\DecValTok{25}\NormalTok{,}\DecValTok{30}\NormalTok{),}\DataTypeTok{Nlim=}\KeywordTok{c}\NormalTok{(}\DecValTok{36}\NormalTok{,}\DecValTok{32}\NormalTok{,}\DecValTok{12}\NormalTok{))}
\CommentTok{#Sum CPUE per net}
\NormalTok{CPUE=}\KeywordTok{data.frame}\NormalTok{(Catch.data.Cod }\OperatorTok
\StringTok{  }\KeywordTok{group_by}\NormalTok{(Mesh_size) }\OperatorTok
\StringTok{  }\KeywordTok{summarize}\NormalTok{(}\DataTypeTok{CPUE_sum =} \KeywordTok{sum}\NormalTok{(CPUE, }\DataTypeTok{na.rm =} \OtherTok{TRUE}\NormalTok{)))}
\CommentTok{#merge to n_lim table}
\NormalTok{n_lim=}\KeywordTok{merge}\NormalTok{(n_lim,CPUE,}\DataTypeTok{by=}\StringTok{"Mesh_size"}\NormalTok{)}
\CommentTok{#Calculate tau (table 25)}
\NormalTok{n_lim}\OperatorTok{$}\NormalTok{tau=n_hours}\OperatorTok{/}\NormalTok{(}\OperatorTok{-}\KeywordTok{log}\NormalTok{(}\DecValTok{1}\OperatorTok{-}\NormalTok{(n_lim}\OperatorTok{$}\NormalTok{CPUE}\OperatorTok{/}\NormalTok{n_lim}\OperatorTok{$}\NormalTok{Nlim)))}
\NormalTok{n_lim}\OperatorTok{$}\NormalTok{N_AP=((n_lim}\OperatorTok{$}\NormalTok{Nlim}\OperatorTok{*}\NormalTok{(}\FloatTok{1.71}\NormalTok{))}\OperatorTok{/}\NormalTok{n_lim}\OperatorTok{$}\NormalTok{tau)}\OperatorTok{*}\NormalTok{n_hours}
\KeywordTok{print}\NormalTok{(n_lim)}
\end{Highlighting}
\end{Shaded}

\begin{verbatim}
##   Mesh_size Nlim CPUE_sum       tau     N_AP
## 1        20   36 25.09091 10.050904 73.49787
## 2        25   32 26.27273  6.974731 94.14557
## 3        30   12 11.72727  3.171088 77.65157
\end{verbatim}

\paragraph{Step 7.3: Size specific
N\_AP}\label{step-7.3-size-specific-n_ap}

!!!In table 26 you use data from table 24 (SL)- is SL=0 if there is no
value for this length group in table 24?

\begin{Shaded}
\begin{Highlighting}[]
\CommentTok{#Replace NA's in column 'SL_l_t' with 0}
\NormalTok{Catch.data.Cod[}\KeywordTok{is.na}\NormalTok{(Catch.data.Cod}\OperatorTok{$}\NormalTok{SL_l_t),}\StringTok{"SL_l_t"}\NormalTok{]=}\DecValTok{0}
\CommentTok{#merge }
\NormalTok{Catch.data.Cod=}\KeywordTok{merge}\NormalTok{(Catch.data.Cod,n_lim,}\DataTypeTok{by=}\StringTok{"Mesh_size"}\NormalTok{)}
\CommentTok{#Calculate N_AP per size class}
\NormalTok{Catch.data.Cod}\OperatorTok{$}\NormalTok{N_AP_size=(Catch.data.Cod}\OperatorTok{$}\NormalTok{N_AP}\OperatorTok{*}\NormalTok{(Catch.data.Cod}\OperatorTok{$}\NormalTok{CPUE}\OperatorTok{/}\NormalTok{Catch.data.Cod}\OperatorTok{$}\NormalTok{CPUE_sum))}\OperatorTok{+}\NormalTok{Catch.data.Cod}\OperatorTok{$}\NormalTok{SL_l_t}
\KeywordTok{print}\NormalTok{(}\KeywordTok{head}\NormalTok{(Catch.data.Cod))}
\end{Highlighting}
\end{Shaded}

\begin{verbatim}
##   Mesh_size Length_group Wedged Tangled Jaw_length        h  y Radian
## 1        20           14      2       2      17.69 4.446094 60   1.08
## 2        20           16     20       5      19.73 4.958815 60   1.08
## 3        20           18     15      12      21.77 5.471535 60   1.08
## 4        20           20     11       8      23.81 5.984256 60   1.08
## 5        20           22     20      18      25.85 6.496977 60   1.08
## 6        20           24     32      24      27.89 7.009698 60   1.08
##        sin       cos       tan       POC      PThC  PWE min_size max_size
## 1 0.514136 0.8577087 0.5994296 0.5488078 0.4511922 0.35       13       31
## 2 0.514136 0.8577087 0.5994296 0.5054167 0.4945833 0.69       13       31
## 3 0.514136 0.8577087 0.5994296 0.4638123 0.5361877 0.81       13       31
## 4 0.514136 0.8577087 0.5994296 0.4239946 0.5760054 0.88       13       31
## 5 0.514136 0.8577087 0.5994296 0.3859636 0.6140364 0.88       13       31
## 6 0.514136 0.8577087 0.5994296 0.3497193 0.6502807 1.00       13       31
##         slope intercept       PEO    Ntotal      CPUE Ntotal_per_hour
## 1 -0.05555556  1.722222 0.9444444  11.02466 0.3636364       0.9187218
## 2 -0.05555556  1.722222 0.8333333  68.81967 2.2727273       5.7349724
## 3 -0.05555556  1.722222 0.7222222  55.28320 2.4545455       4.6069330
## 4 -0.05555556  1.722222 0.6111111  48.24247 1.7272727       4.0202057
## 5 -0.05555556  1.722222 0.5000000 117.76900 3.4545455       9.8140830
## 6 -0.05555556  1.722222 0.3888889 235.29074 5.0909091      19.6075616
##       SL_l_t Nlim CPUE_sum     tau     N_AP N_AP_size
## 1  0.5550854   36 25.09091 10.0509 73.49787  1.620272
## 2  3.4622451   36 25.09091 10.0509 73.49787 10.119661
## 3  2.1523876   36 25.09091 10.0509 73.49787  9.342396
## 4  2.2929329   36 25.09091 10.0509 73.49787  7.352569
## 5  6.3595376   36 25.09091 10.0509 73.49787 16.478809
## 6 14.5166525   36 25.09091 10.0509 73.49787 29.429263
\end{verbatim}

\subsubsection{Step 8: Calculate
selectivity}\label{step-8-calculate-selectivity}

(table 28)

\begin{Shaded}
\begin{Highlighting}[]
\NormalTok{##Calculate selectivity}
\NormalTok{Catch.data.Cod}\OperatorTok{$}\NormalTok{selectivity=Catch.data.Cod}\OperatorTok{$}\NormalTok{CPUE}\OperatorTok{/}\NormalTok{Catch.data.Cod}\OperatorTok{$}\NormalTok{Ntotal_per_hour}
\NormalTok{##plot}
\NormalTok{Catch.data.Cod}\OperatorTok{$}\NormalTok{Mesh_size_fac=}\KeywordTok{as.character}\NormalTok{(Catch.data.Cod}\OperatorTok{$}\NormalTok{Mesh_size)}
\KeywordTok{ggplot}\NormalTok{(Catch.data.Cod, }\KeywordTok{aes}\NormalTok{(}\DataTypeTok{x=}\NormalTok{Length_group, }\DataTypeTok{y=}\NormalTok{selectivity, }\DataTypeTok{group=}\NormalTok{Mesh_size_fac)) }\OperatorTok{+}
\StringTok{  }\KeywordTok{geom_line}\NormalTok{(}\KeywordTok{aes}\NormalTok{(}\DataTypeTok{color=}\NormalTok{Mesh_size_fac))}\OperatorTok{+}
\StringTok{  }\KeywordTok{geom_point}\NormalTok{(}\KeywordTok{aes}\NormalTok{(}\DataTypeTok{color=}\NormalTok{Mesh_size_fac))}\OperatorTok{+}
\StringTok{   }\KeywordTok{labs}\NormalTok{(}\DataTypeTok{x =} \StringTok{"length Group"}\NormalTok{,}\DataTypeTok{y=}\StringTok{"Selectivity"}\NormalTok{,}\DataTypeTok{color=}\StringTok{"Mesh Size"}\NormalTok{)}
\end{Highlighting}
\end{Shaded}

\begin{verbatim}
## Warning: Removed 25 rows containing missing values (geom_path).
\end{verbatim}

\begin{verbatim}
## Warning: Removed 25 rows containing missing values (geom_point).
\end{verbatim}

\includegraphics{Lobyrev_V2_no_rounding_files/figure-latex/unnamed-chunk-16-1.pdf}


\end{document}
